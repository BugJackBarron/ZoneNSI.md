%%!TEX encoding = UTF-8 Unicode
\documentclass[a4paper,12pt]{article}
\input{../../preambule.tex}
\input{../../figures.tex}



\lhead{\ccby}
\rhead{\small{ $2^{\text{nde}}$}}
\chead{\small{ SNT - Web}}


\lfoot{\tiny{Ann\'ee 2019-2020}}
\cfoot{\textbf{Page \thepage/\pageref{LastPage}}}
%\cfoot{}
\rfoot{\tiny{www.mathtice.org}}


\begin{document}
\sffamily %Pourutiliser une police sans empatement
\begin{center}
 \large{\textbf{$N3$ Construire une page Web : HTML}}
\end{center}


\section{Une page web ?}
\begin{enumerate}
\item Dans le lecteur réseau de votre classe, dossier \texttt{Documents en consultation/SNT/Web}, récupérez le fichier \texttt{N3\_Micro\_Maqueen} et copiez-le dans votre répertoire \texttt{SNT} dans lequel vous aurez au préalable créé un dossier \texttt{web}.
\item Double-cliquez sur ce fichier. Quel logiciel l'ouvre ? \caserepsanssaut{6}{0.7}
\item Quelle est la nature du contenu du fichier ? \caserepsanssaut{6}{0.7}
\item Vous pouvez fermer le logiciel. Nous allons changer l'extension du fichier \texttt{N3\_Micro\_Maqueen}. Pour ce faire :
\begin{enumerate}
\item Placez vous dans votre dossier \texttt{SNT/web} à l'aide de votre explorateur de fichier,  puis cliquez sur le menu en haut \texttt{Affichage}, puis cochez la case \texttt{Extensions de noms de fichiers}. Quelle extension a pour l'instant le fichier \texttt{N3\_Micro\_Maqueen} ?
\caserepsanssaut{4}{0.7}
\item Sélectionnez le fichier puis appuyez sur \texttt{F2} pour renommer, et changez l'extension \texttt{.txt} en \texttt{.html}. Que se passe-t-il pour l'icône de fichier ? \caserepsanssaut{6}{0.7}
 \end{enumerate}
\item Double-cliquez sur le fichier. Quel logiciel l'ouvre ? \caserepsanssaut{6}{0.7}
\item Quelle est la nature du contenu du fichier ? \caserepsanssaut{6}{0.7}
\item Quelle est l'URL de la page ? \caserepsanssaut{10}{0.7}
\item Dans le dossier \texttt{Documents en consultation/SNT/Web}, récupérez les fichiers images\\ \texttt{BBC\_Microbit.jpg} et \texttt{\_Maqueen.png}, et copiez les dans le même répertoire que votre fichier \texttt{N3\_Micro\_Maqueen.html}. Rechargez la page web.\\ Quels sont les changements apportés ? \caserepsanssaut{6}{0.7}
\item Réduisez sans fermer votre navigateur, puis cliquez-droit sur le fichier \texttt{N3\_Micro\_Maqueen.html}, et sélectionnez \texttt{Edit with Notepad++}. Que retrouvez-vous ? \caserepsanssaut{6}{0.7}
\item Quelles différences constatez-vous entre le fichier ouvert dans un navigateur, et le fichier ouvert par \texttt{Notepad++}?
\caserep{\linewidth}{6}
\end{enumerate}
\pagebreak
\section{Le fond : du texte et de l'hypertexte - HTML}
\begin{info}{Ce qu'est une page Web}
\noindent Une page web, c'est un fichier contenant du texte, ce texte étant enrichi par un système de \textbf{balises} ouvrantes - par exemple \texttt{<p>} et fermantes - par exemple \texttt{</p>}, permettant de donner un sens particulier au texte encadré par ces balises.

\noindent Ces balises sont interprétées par le navigateur web et donnent une forme particulière au texte, ou bien introduisent des comportement particuliers à certains éléments ( transforment en lien hypertextuels, insèrent des images, etc).

\noindent Le langage utilisé qui contient le texte et les balises s'appelle \textbf{HTML} ( d'où le nom de l'extension \texttt{.html} ) , qui est un acronyme pour \textbf{Hyper Text Markup Language}, soit langage de balisage hypertexte. Il est à noter que HTML \textbf{n'est pas un langage de programmation}, mais simplement un langage de mise en valeur du texte, privilégiant la mise en avant du sens avant la forme.\\

\noindent Il existe de nombreux langages de balisages, permettant de donner une forme particulière à du texte ou à des chaines de caractères. Par exemple, HTML pour le Web, \LaTeX pour les documents scientifiques et les formules mathématiques, XML pour organiser des données, ou même Mardown, un langage ultra simplifié pour créer des Notebooks Jupyter ( que nous utiliserons plus tard dans l'année ).
\end{info}

\noindent L'objectif des questions suivantes est de vous faire comprendre le rôle des balises HTML. Vous devez donc comparer les deux visions du fichier HTML, celle vue par le navigateur et celle vue par l'éditeur de texte.
\begin{enumerate}
\item Prenons la deuxième ligne du fichier, vue dans \texttt{Notepad++}. Celle ci contient une \textbf{balise ouvrante} de \textbf{nom} \texttt{html}, et d'\textbf{attribut} \texttt{lang}, encadrés par des \textbf{chevrons} < et >.
\begin{enumerate}
\item L'attribut \texttt{lang} possède une valeur. Quelle est-elle ? \caserepsanssaut {6}{0.7}
\item Toute balise ouvrante doit être fermée. Où se trouve la balise fermante correspondant à \texttt{<html lang="fr">} ? Avec quoi la distingue-t-on ? \caserepsanssaut {6}{0.7}
\end{enumerate}
\item Pour chacun des cas ci-dessous, trouver le couple de balise qui correspond :
\begin{enumerate}
\item Texte affiché dans l'onglet : \caserepsanssaut {6}{0.5}
\item Titre principal : \caserepsanssaut {6}{0.5}
\item Sous-titre : \caserepsanssaut {6}{0.5}
\item Sous-sous-titre : \caserepsanssaut {6}{0.5}
\item Paragraphe : \caserepsanssaut {6}{0.5}
\item Mise en gras dans un paragraphe : \caserepsanssaut {6}{0.5}
\item Hyperlien :  \caserepsanssaut {6}{0.5}
\item Liste numérotée : \caserepsanssaut {6}{0.5}
\item Liste à puces : \caserepsanssaut {6}{0.5}
\item Haut de page : \caserepsanssaut {6}{0.5}
\item Pied de page : \caserepsanssaut {6}{0.5}
\item Corps principal de la page : \caserepsanssaut {6}{0.5}
\item En-tête de la page : \caserepsanssaut {6}{0.5}
\item Ensemble du code html : \caserepsanssaut {6}{0.5}
\end{enumerate}
\item \begin{enumerate}
\item Concernant les liens hypertextes de la page, les deux ne se comportent pas de la même manière. Comment expliquer cela ?
\caserep{\linewidth}{2}
\item Toujours concernant les liens hypertextes, que doit-on renseigner pour donner la direction du lien ?\caserep{\linewidth}{2}
\end{enumerate}
\item \begin{enumerate}
\item Nous n'avons pas évoqué les images présentes sur la page. Quelle balise permet d'afficher ces images ? Qu'a-t-elle de particulier par rapport à toutes celles vues avant ?
\caserep{\linewidth}{3}
\item Faites survoler ces deux images par votre pointeur de souris dans votre navigateur. Vous devez constater une différence de traitement entre les deux. Comment l'expliquer ?
\caserep{\linewidth}{3}
\item Mais, au départ, avant de copier les fichiers images dans le même répertoire que le fichier \texttt{N3\_Micro\_Maqueen.html}, les images n'étaient pas affichées. Par quoi étaient elles remplacées ?
\caserep{\linewidth}{3}

\end{enumerate}
\item Le pied de page est particulier. Qu'a-t-il de  spécial ?
\caserep{\linewidth}{2}
\item Il y a trois vidéos, toutes issues de \texttt{Youtube}
\begin{enumerate}
\item Quelle balise permet de les visionner ? Est-elle fermante ? 
\caserep{\linewidth}{1}
\item Quels sont les attributs donnés, et à quoi correspondent-ils ?
\caserep{\linewidth}{4}
\item L'une de ces vidéos n'a pas le même comportement que les deux autres. Comment expliquer cela ?
\caserep{\linewidth}{1}
\begin{info}{Trouver le code pour insérer une vidéo}
\noindent Quand vous visionnez une vidéo sur \texttt{Youtube}, sous la vidéo vous avez un icône \texttt{Partager}. En cliquant sur cet icône, vous obtenez des liens pour partager sur \texttt{Facebook}, \texttt{Twitter}, etc. Mais vous avez aussi un icône \texttt{Intégrer}. En cliquant sur celui-ci, vous obtenez la balise complète vous permettant d'intégrer la vidéo en question dans votre site web.
\end{info}
\end{enumerate}
\item Au milieu du code est inséré ce qu'on appelle un commentaire, c'est-à-dire une ligne qui ne peut être lue que par ceux qui regardent le code source ( le contenu réel du HTML ). Cette ligne sert à faire des commentaires à la personne qui voudra comprendre le code donné. Quelle balise introduit ces commentaires ?
\caserep{\linewidth}{1}
\end{enumerate}
\section{\`A vous de jouer}
\noindent Vous devez créer une page web sur le sujet de votre choix, contenant un titre principal, deux sous-parties, au moins deux images et une vidéo.
\end{document}
