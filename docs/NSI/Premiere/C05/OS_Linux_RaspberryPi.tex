%%!TEX encoding = UTF-8 Unicode

\documentclass[a4paper,12pt]{article}
\input{../../preambule.tex}
\input{../../figures.tex}



\lhead{\ccby}
\rhead{\small{ $1^{\text{ère}}$ NSI}}
\chead{\small{ $C06$ Architecture matérielle et systèmes d'exploitation}}


\lfoot{\tiny{Ann\'ee 2020-2021}}
\cfoot{\textbf{Page \thepage/\pageref{LastPage}}}
%\cfoot{}
\rfoot{\tiny{www.zonensi.fr}}

\begin{document}
\sffamily %Pourutiliser une police sans empatement
\begin{center}
 \large{\textbf{$C06-01$ Architecture Matérielle}}
\end{center}


https://www.clubic.com/raspberry-pi/article-850381-1-installation-os-raspberry-pi-utilitaire-noobs.html

https://github.com/raspberrypi/noobs

https://dadarevue.com/ajouter-gui-raspbian-lite/

\begin{enumerate}
\item La carte est formattée, et Noobs est installé
\item Installer Raspian 32bits sans interface graphique
\item Mise à jour possible si wifi configuré dans Noobs
\item Login : pi MDP : raspberry
\item sudo halt et sudo reboot


\end{enumerate}
Installation de l'environnement graphique

\begin{enumerate}
\item sudo apt-get update
sudo apt-get dist-upgrade
sudo reboot
\item sudo apt-get install --no-install-recommends xserver-xorg
\item sudo apt-get install raspberrypi-ui-mods
\item sudo apt-get install lightdm
\item sudo reboot
\end{enumerate}

\end{document}
